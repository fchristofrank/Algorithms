\documentclass[11pt]{article}

\newcommand{\yourname}{}

\def\comments{0}
\def\code#1{\texttt{#1}}
\newcommand{\finishproblem}{
    \vspace{10pt}
    \hrule
    \vspace{10pt}
}
%format and packages

%\usepackage{algorithm, algorithmic}
\documentclass{article}
\usepackage{algorithm}
\usepackage{algorithmicx}
\usepackage{algpseudocode}
\usepackage{algpseudocode}
\usepackage{amsmath, amssymb, amsthm}
\usepackage{enumerate}
\usepackage{enumitem}
\usepackage{framed}
\usepackage{verbatim}
\usepackage[margin=1.0in]{geometry}
\usepackage{microtype}
\usepackage{kpfonts}
\usepackage{palatino}
	\DeclareMathAlphabet{\mathtt}{OT1}{cmtt}{m}{n}
	\SetMathAlphabet{\mathtt}{bold}{OT1}{cmtt}{bx}{n}
	\DeclareMathAlphabet{\mathsf}{OT1}{cmss}{m}{n}
	\SetMathAlphabet{\mathsf}{bold}{OT1}{cmss}{bx}{n}
	\renewcommand*\ttdefault{cmtt}
	\renewcommand*\sfdefault{cmss}
	\renewcommand{\baselinestretch}{1.06}
\usepackage[usenames,dvipsnames]{xcolor}
\definecolor{DarkGreen}{rgb}{0.15,0.5,0.15}
\definecolor{DarkRed}{rgb}{0.6,0.2,0.2}
\definecolor{DarkBlue}{rgb}{0.2,0.2,0.6}
\definecolor{DarkPurple}{rgb}{0.4,0.2,0.4}
\usepackage[pdftex]{hyperref}
\hypersetup{
	linktocpage=true,
	colorlinks=true,				% false: boxed links; true: colored links
	linkcolor=DarkBlue,		% color of internal links
	citecolor=DarkBlue,	% color of links to bibliography
	urlcolor=DarkBlue,		% color of external links
}

\usepackage[boxruled,vlined,nofillcomment]{algorithm2e}
	\SetKwProg{Fn}{Function}{\string:}{}
	\SetKwFor{While}{While}{}{}
	\SetKwFor{For}{For}{}{}
	\SetKwIF{If}{ElseIf}{Else}{If}{:}{ElseIf}{Else}{:}
	\SetKw{Return}{Return}
	

%enclosure macros
\newcommand{\paren}[1]{\ensuremath{\left( {#1} \right)}}
\newcommand{\bracket}[1]{\ensuremath{\left\{ {#1} \right\}}}
\renewcommand{\sb}[1]{\ensuremath{\left[ {#1} \right\]}}
\newcommand{\ab}[1]{\ensuremath{\left\langle {#1} \right\rangle}}

%probability macros
\newcommand{\ex}[2]{{\ifx&#1& \mathbb{E} \else \underset{#1}{\mathbb{E}} \fi \left[#2\right]}}
\newcommand{\pr}[2]{{\ifx&#1& \mathbb{P} \else \underset{#1}{\mathbb{P}} \fi \left[#2\right]}}
\newcommand{\var}[2]{{\ifx&#1& \mathrm{Var} \else \underset{#1}{\mathrm{Var}} \fi \left[#2\right]}}

%useful CS macros
\newcommand{\poly}{\mathrm{poly}}
\newcommand{\polylog}{\mathrm{polylog}}
\newcommand{\zo}{\{0,1\}}
\newcommand{\pmo}{\{\pm1\}}
\newcommand{\getsr}{\gets_{\mbox{\tiny R}}}
\newcommand{\card}[1]{\left| #1 \right|}
\newcommand{\set}[1]{\left\{#1\right\}}
\newcommand{\negl}{\mathrm{negl}}
\newcommand{\eps}{\varepsilon}
\DeclareMathOperator*{\argmin}{arg\,min}
\DeclareMathOperator*{\argmax}{arg\,max}
\newcommand{\eqand}{\qquad \textrm{and} \qquad}
\newcommand{\ind}[1]{\mathbb{I}\{#1\}}
\newcommand{\sslash}{\ensuremath{\mathbin{/\mkern-3mu/}}}

%mathbb
\newcommand{\N}{\mathbb{N}}
\newcommand{\R}{\mathbb{R}}
\newcommand{\Z}{\mathbb{Z}}
%mathcal
\newcommand{\cA}{\mathcal{A}}
\newcommand{\cB}{\mathcal{B}}
\newcommand{\cC}{\mathcal{C}}
\newcommand{\cD}{\mathcal{D}}
\newcommand{\cE}{\mathcal{E}}
\newcommand{\cF}{\mathcal{F}}
\newcommand{\cL}{\mathcal{L}}
\newcommand{\cM}{\mathcal{M}}
\newcommand{\cO}{\mathcal{O}}
\newcommand{\cP}{\mathcal{P}}
\newcommand{\cQ}{\mathcal{Q}}
\newcommand{\cR}{\mathcal{R}}
\newcommand{\cS}{\mathcal{S}}
\newcommand{\cU}{\mathcal{U}}
\newcommand{\cV}{\mathcal{V}}
\newcommand{\cW}{\mathcal{W}}
\newcommand{\cX}{\mathcal{X}}
\newcommand{\cY}{\mathcal{Y}}
\newcommand{\cZ}{\mathcal{Z}}

%theorem macros
\newtheorem{thm}{Theorem}
\newtheorem{lem}[thm]{Lemma}
\newtheorem{fact}[thm]{Fact}
\newtheorem{clm}[thm]{Claim}
\newtheorem{rem}[thm]{Remark}
\newtheorem{coro}[thm]{Corollary}
\newtheorem{prop}[thm]{Proposition}
\newtheorem{conj}[thm]{Conjecture}

\theoremstyle{definition}
\newtheorem{defn}[thm]{Definition}


\newcommand{\instructor}{Iraklis Tsekourakis}
\newcommand{\hwnum}{1}
%\newcommand{\hwdue}{Wednesday, January 27 at 11:59pm via \href{https://gradescope.com/courses/229309}{Gradescope}}

\newtheorem{prob}{}
\newtheorem{sol}{Solution}

\definecolor{cit}{rgb}{0.05,0.2,0.45} 
\newcommand{\solution}{\medskip\noindent{\color{DarkBlue}\textbf{Solution:}}}


\begin{document}
{\Large 
\begin{center}{CS5800: Algorithms} ---  \instructor \end{center}}
{\large
\vspace{10pt}
\noindent Homework~\hwnum \vspace{2pt}%\\
%Due :~\hwdue
}

\bigskip
{\large \noindent Name: Christo Frank Franklin }


\vspace{15pt}

%%%%%%%%%%%%%%%%%%%%%%%%%%%%%%%%%%%%%%%%%%%%%%%%%%%%%%%%
%%%%%%%%%%%%%%%%%%%%%%%%%%%%%%%%%%%%%%%%%%%%%%%%%%%%%%%% Problem 1
\begin{prob} \textbf{(18 points)} In the following, use a direct proof (by giving values for $c$ and $n_0$ in the formal definition of big-$O/\Omega$ notation) to prove that:
\end{prob}
\begin{enumerate}[label=(\alph*)]

%%%%%%%%%%%%%%%%%%%%%%%%%%%%%%%%%%% 1a
\item $n^{2}+7n+1 \text { is } \Omega(n^{2})$ \\
\solution
\\

The given function, \( f(n) = n^2 + 7n + 1 \).  

\textbf{By the formal definition of the lower bound (Big-Omega notation), there exist positive constants \( c \) and \( n_0 \) such that for all \( n \geq n_0 \):}  

\[
f(n) \geq c \cdot g(n)
\]

\[
n^2 + 7n + 1 \geq c \cdot g(n)
\]

Considering \( g(n) = n^2 \), we have  

\[
n^2 + 7n + 1 \geq c \cdot n^2
\]

Taking the value of \( c = 1 \), we get  

\[
n^2 + 7n + 1 \geq n^2
\]

\[
7n + 1 \geq 0
\]

\[
n \geq -\frac{1}{7}
\]

Since we consider asymptotic behavior for large values of \( n \), we take \( n_0 = 1 \).  
Thus, for all \( n \geq 1 \) and with \( c = 1 \), the inequality holds. 

\[
\therefore f(n) = \Omega(n^2)
\]


%%%%%%%%%%%%%%%%%%%%%%%%%%%%%%%%%%% 1b
\item $3n^{2}+n-10 \text { is } O(n^{2})$ \\
\solution

The given function, \( f(n) = 3n^2 + n - 10 \).\\
\textbf{By the formal definition of Big O:}\\
Let \( f(n) = O(g(n)) \) if and only if there exist positive constants \( c \) and \( n_0 \) such that for all \( n \geq n_0 \),
\[
f(n) \leq c \cdot g(n)
\]
\textbf{Now applying this to the problem:}\\
If \( f(n) \) is \( O(n^2) \), then  
\[
f(n) \leq c \cdot g(n)
\]
\[
3n^2 + n - 10 \leq c \cdot g(n)
\]
Considering \( g(n) = n^2 \), we have  
\[
3n^2 + n - 10 \leq c \cdot n^2
\]
Taking the value of \( c = 4 \), we get  
\[
3n^2 + n - 10 \leq 4n^2
\]
\[
n^2 - n + 10 \geq 0
\]
From this expression, for any value of \( n \), the equation is satisfied.\\
Hence, for all values above \( n > 1 \), here for \( n = 1 \) and with \( c = 4 \), the equation holds good.\\

\textbf{Now let's check for a particular value:}\\
We check if the inequality holds true for \( n = 2 \) and \( c = 4 \).\\

For \( f(n) = 3n^2 + n - 10 \) and \( g(n) = n^2 \), we check if:
\[
f(n) \leq c \cdot g(n)
\]
Substituting \( n = 2 \) and \( c = 4 \), we get:
\[
3(2)^2 + 2 - 10 \leq 4 \cdot (2)^2
\]
\[
3(4) + 2 - 10 \leq 4 \cdot 4
\]
\[
12 + 2 - 10 \leq 16
\]
\[
4 \leq 16
\]
This inequality holds true for \( n = 2 \) and \( c = 4 \).
\\ \\ \\ \\
%%%%%%%%%%%%%%%%%%%%%%%%%%%%%%%%%%% 1c
\item $n^2 \text { is } \Omega(n\lg n)$ \\
\solution

\textbf{Given function:} \( f(n) = n^2 \)

\textbf{To Prove:} \( f(n) \) is \( \Omega(n^2) \) by the formal definition.

\textbf{By the formal definition of Big Omega:}\\
Let \( f(n) = \Omega(g(n)) \) if and only if there exist positive constants \( c \) and \( n_0 \) such that for all \( n \geq n_0 \),
\[
f(n) \geq c \cdot g(n)
\]

\textbf{Now applying this to the problem:}\\
If \( f(n) \) is \( \Omega(n^2) \), then
\[
f(n) \geq c \cdot g(n)
\]
\[
n^2 \geq c \cdot g(n)
\]

\textbf{Considering:} 
\[
g(n) = n \log n
\]
we have
\[
n^2 \geq c \cdot n \log n
\]

\textbf{Dividing both sides by} \( n \log n \):
\[
\frac{n}{\log n} \geq c
\]
\[
c \leq \frac{n}{\log n}
\]

\textbf{Verification:}\\
For \( c = 4 \) and \( n = 10 \), the inequality holds true:
\[
\frac{10}{\log 10} = \frac{10}{2.3026} \approx 4.34 \geq 4
\]

Thus, the inequality holds for \( n = 10 \) for \( c = 4 \).

\end{enumerate}
\\
\\
\\
\\
\\
%%%%%%%%%%%%%%%%%%%%%%%%%%%%%%%%%%%%%%%%%%%%%%%%%%%%%%%%
%%%%%%%%%%%%%%%%%%%%%%%%%%%%%%%%%%%%%%%%%%%%%%%%%%%%%%%% Problem 2
\begin{prob} \textbf{(20 points)} In the following, use the iteration method to find the asymptotic notation of the order of growth of the recurrences:
\end{prob}

\begin{enumerate}[label=(\alph*)]
%%%%%%%%%%%%%%%%%%%%%%%%%%%%%%%%%%% 2a
\item $T(n)= \begin{cases}1 & \text { if } n=1 \\
2 T\left(\frac{n}{2}\right)+b & \text { if } n>1\end{cases}$ \\
\solution

\textbf{Given Recurrence function is}
\[
T(n) = 2 \cdot T(n/2) + b
\]
Upon next iteration,
\[\quad T(n) = 2 \cdot \left( 2 \cdot \left( T(n/4) + b \right) \right) + b\]
\[\quad T(n) = 4 \cdot T(n/4) + 3b\]

Upon next iteration, Substituting for \( T(n/4) \),  

\[
\quad T(n) = 4 \cdot \left( 2 \cdot T(n/8) + b \right) + 3b
\]
\[
\quad T(n) = 8 \cdot T(n/8) + 7b
\]

If \( n = k \),  
\[
T(k) = 2^k \cdot T(k/2^k) + (2^k - 1) \cdot b
\]

Upon subsequent substitution, it reaches the base case \( T(1) \) when  
\[ \quad \frac{k}{2^k} = 1 \]
\[ \quad k = \log n \]
\[ T(n) = 2^{\log n} \cdot T(1) + b \cdot \left( 2^{\log n} - 1 \right) \]
\[\quad = n + b \cdot (n - 1)\]
\[ \quad = n(1 + b) - b \]

Hence, the asymptotic notation would be \( \theta(n) \).

%%%%%%%%%%%%%%%%%%%%%%%%%%%%%%%%%%% 2b
\item $T(n)= \begin{cases}c & \text { if } n=0 \\
T(n-1)+n+b & \text { if } n>1\end{cases}$ \\
\solution

\[
T(n) = T(n-1) + n + b
\]
\[
T(n) = \left[ T(n-2) + (n-1) + b \right] + n + b
\]
\[
T(n) = \left[ T(n-3) + (n-2) + b \right] + (n-1) + b + n + b
\]
\[
T(n) = T(n-3) + (n-2) + (n-1) + n + 3b
\]

\[
T(n) = T(n-k) + (n-k) + \dots + (n-2) + (n-1) + n + 3b
\]

\[
T(n) = T(0) + \sum_{i=0}^{n} (n-i) + nb
\]

\[
= \frac{n(n+1)}{2} + nb + c
\]

\[
= \frac{n^2}{2} + \frac{n}{2} + nb + c
\]

\text{From this, it can be inferred that the growth of complexity is } \[ \theta(n^2) \]

\finishproblem
\end{enumerate}


%%%%%%%%%%%%%%%%%%%%%%%%%%%%%%%%%%%%%%%%%%%%%%%%%%%%%%%%
%%%%%%%%%%%%%%%%%%%%%%%%%%%%%%%%%%%%%%%%%%%%%%%%%%%%%%%% Problem 3
\begin{prob} \textbf{(20 points)} Solve the following recurrences using the substitution method:
\end{prob}

\begin{enumerate}[label=(\alph*)]
%%%%%%%%%%%%%%%%%%%%%%%%%%%%%%%%%%% 3a
\item $T(n)= T(n-3) + 3\lg n$. Our guess is $T(n) = O(nlgn)$. Show that$ T(n) \le cn\lg n$ for some constant $c > 0$ (Note that $\lg n$ is monotonically increasing for $n>0$)\\
\solution

\textbf{Our guess:} \( O(n \log n) \)  

\textbf{By formal definition,}  
\[
T(n) \leq C \cdot n \log n \quad \text{for constant } C > 0
\]

\[
T(n) \leq C(n-3) \log(n-3) + 3 \log n
\]

\textbf{Since the logarithm function is monotonically increasing,}  
\[
\log n \geq \log (n-3)
\]

\[
T(n) \leq C(n-3) \log n + 3 \log n
\]

\[
T(n) \leq C n \log n - 3C \log n + 3 \log n
\]

\[
T(n) \leq C n \log n - (3C + 3) \log n
\]

\textbf{For all } \( C \geq 1 \), \text{since the second term must be negative, }  
\[
3C + 3 \leq 0 \quad \Rightarrow \quad C \geq 1
\]

\[
T(n) \leq C n \log n - (3C + 3) \log n
\]

\textbf{To verify, Let consider } \( C = 1 \), \text{ the equation holds true:}  
\[
T(n) \leq n \log n - 6 \log n
\]

\textbf{Hence, the complexity of } \( T(n) \) \text{ is } \( O(n \log n) \).
\\

\finishproblem
\end{enumerate}

%%%%%%%%%%%%%%%%%%%%%%%%%%%%%%%%%%% 3b
\begin{enumerate}[label=(b)]
\item $T(n)= 4T(n/3) + n$. Our guess is $T(n) = O(n^{\log_3 4})$. Show that $T(n) \le cn^{\log_3 4}$ for some constant $c > 0$\\
\solution

\textbf{Our guess is} \[ O(n^{\log_3 4}) \]

\textbf{By Formal Definition,}

\[
T(n) \leq C \cdot n^{\log_3 4} \quad \text{for constant } C > 0
\]

\[
T(n) \leq 4 \cdot \left( C \cdot \left( \frac{n}{3} \right)^{\log_3 4} \right) + n
\]

\[
T(n) \leq 4C \cdot \frac{n^{\log_3 4}}{4} + n
\]

\[
T(n) \leq C \cdot n^{\log_3 4} + n
\]

\text{The } n \text{ term:}

\[
T(n) \leq C \cdot n^{\log_3 4} + n
\]

\text{Since we have an additional linear term.}

\textbf{Improving the guess}

\[
T(n) \leq C \cdot n^{\log_3 4} - d n
\]

\textbf{Substituting the guess to the given function}

\[
T(n) \leq 4 \bracket{ C \cdot \left(\frac{n}{3}\right)^{\log_3 4} - d (n/3) }+ n
\]

\[
T(n) \leq 4C \cdot \left(\frac{n^{\log_3 4}}{3^{\log_3 4}}\right) - 4 d \cdot (n/3) + n
\]

\[
T(n) \leq 4C \cdot \left(\frac{n^{\log_3 4}}{4}\right) - 4d \cdot (n/3) + n
\]

\text{For } C = 1 and d = 1:

\[
T(n) \leq n^{\log_3 4} - (4n/3) + n
\]

\[
T(n) \leq n^{\log_3 4} - (n/3)
\]

\textbf{Hence, the complexity would be O (n^{\log_3 4} - n)}

\end{enumerate}

\\
\\
\finishproblem
%%%%%%%%%%%%%%%%%%%%%%%%%%%%%%%%%%%%%%%%%%%%%%%%%%%%%%%%
%%%%%%%%%%%%%%%%%%%%%%%%%%%%%%%%%%%%%%%%%%%%%%%%%%%%%%%% Problem 4
\begin{prob} \textbf{(20 points)} You can also think of insertion sort as a recursive algorithm. In order to sort \code{A[1:n]}, recursively sort the subarray \code{A[1:n-1]}. Write pseudocode for this recursive version of insertion sort. Give a recurrence for its worst-case running time.
\end{prob}
\solution

\textbf{Algorithm: InsertionSort} 

\begin{verbatim}
InsertionSort(arr, n):
    if n < 1:
        return

    InsertionSort(arr, n-1)

    key = arr[n]
    index = n - 1

    while index > 0 and key < arr[index]:
        arr[index + 1] = arr[index]
        index -= 1

    arr[index + 1] = key

\end{verbatim}
\begin{enumerate}

To calculate the recurrence relation for the worst case,

\[
T(n) = 
\begin{cases} 
    c & \text{if } n = 1 \text{ (base case)}\\
    T(n-1) + c(n-1) + d & \text{if } n > 0 
\end{cases}
\]

By using iteration method,\\

1st Iteration : \[ T(n) = T(n-1) + c(n-1) + d \]

2nd Iteration : \[ T(n) = T(n-2) + c(n-2) + d + c(n-1) + d \]

3rd Iteration : \[ T(n) = T(n-3) + c(n-3) + d c(n-2) + d + c(n-1) + d + \]

(n-1)th Iteration : \[ T(n) = T(1) + c(n-(n-1))+ c(n-(n-2)) + ... + (n-1).d \]

\[ T(n) = T(1) + c(1+2+3+4..+(n-1)) + ... + (n-1).d \]

\[ T(n) = T(1) + c(\frac{n(n-1)}{2}) + (n-1).d \]

\[ T(n) = k + c(\frac{n(n-1)}{2}) + (nd-d) \]

\[ T(n) = k + \frac{cn^2}{2} - \frac{cn}{2} + (nd-d) \]

We combine, k, $-\frac{cn}{2}$ , -d to a single term to parse the logic better

\[ T(n) = \frac{cn^2}{2} + nd + a \]

where a = $k - \frac{cn}{2} -d$ \\

Therefore, the expression we can understand that the term $n^2$, which increases faster than the \\
other linear terms. Thus proving that $T(n) = O(n^2)$


\end{enumerate}

\finishproblem
%%%%%%%%%%%%%%%%%%%%%%%%%%%%%%%%%%%%%%%%%%%%%%%%%%%%%%%%
%%%%%%%%%%%%%%%%%%%%%%%%%%%%%%%%%%%%%%%%%%%%%%%%%%%%%%%% Problem 5
\begin{prob} \textbf{(20 points)} Let $f(n)$ and $g(n)$ be asymptotically nonnegative functions. Using the basic definition of $\Theta$-notation, prove that $\max{f(n), g(n)} = \Theta(f(n)+g(n))$
\end{prob}
\solution

\textbf{By Definition of} \( \Theta(f(n) + g(n)) \),

\[
\max \{ f(n), g(n) \} = \Theta(f(n) + g(n))
\]

\[
c_1(f(n) + g(n)) \leq \max(f(n), g(n)) \leq c_2(f(n) + g(n))
\]

\textbf{Considering the Upper Bound,}

\[
\max(f(n), g(n)) \leq c_2(f(n) + g(n))
\]

For \( c_2 \geq 1 \), then

\[
\max(f(n), g(n)) \leq f(n) + g(n)
\]

\textbf{Considering the Lower Bound,}

\[
\max(f(n), g(n)) \geq c_1(f(n) + g(n))
\]

For \( c_1 \leq \frac{1}{2} \), then

\[
\max(f(n), g(n)) \geq 0.5(f(n) + g(n))
\]

\finishproblem

%%%%%%%%%%%%%%%%%%%%%%%%%%%%%%%%%%%%%%%%%%%%%%%%%%%%%%%%
%%%%%%%%%%%%%%%%%%%%%%%%%%%%%%%%%%%%%%%%%%%%%%%%%%%%%%%% Problem 6
\begin{prob} \textbf{(20 points)} Is $2^{n+1}=O(2^n)$? Is $2^{2n}=O(2^n)$? Use the formal definition of $O$-notation to answer these two questions.
\end{prob}
\solution

\textbf{(a)} 

Given function is \( f(n) = 2^{(n+1)} \). \\
Applying the formal definition:

\[
f(n) \leq c \cdot g(n)
\]

Let's consider \( g(n) = 2^n \),

\[
2^{(n+1)} \leq C \cdot 2^n
\]

If \( c \geq 2 \):

Taking \( c = 2 \),

\[
2^{(n+1)} \leq 2 \cdot 2^n
\]

\[
2^{(n+1)} \leq 2^{(n+1)}
\]

For \( c \geq 2 \), taking \( c = 4 \),

\[
2^{(n+1)} \leq 4 \cdot 2^n
\]

\[
2^{(n+1)} \leq 2^2 \cdot 2^n
\]

\[
2^{(n+1)} \leq 2^{(2 + n)}
\]

\t \t The equation is satisfied for the value of c.


\textbf{(b)}

Given:

\[
f(n) = 2^{(2n)}
\]
\[
g(n) = 2^n
\]

Applying the formal definition:

\[
f(n) \leq c \cdot g(n)
\]

\[
2^{(2n)} \leq c \cdot 2^n
\]

\[
c \geq 2^n
\]

There cannot be any constant \( c \) such that it is always greater than \( 2^n \), 
as \( 2^n \) grows exponentially.

\end{enumerate}

\end{document}
